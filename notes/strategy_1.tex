\documentclass[11pt]{article}

% ----------------------------------------------------------- essential packages
\usepackage[margin=1in]{geometry}
\usepackage{amsmath}
%\usepackage{amsfonts}
\usepackage{amsthm}
\usepackage{amssymb}
\usepackage{booktabs} % somewhat essential, I guess
\usepackage{setspace} % for double spacing, etc.

% ------------------------------------------------------------------- font stuff
\renewcommand{\rmdefault}{ptm}
\usepackage[lite,zswash,eucal]{mtpro2}
%\usepackage[charter]{mathdesign}
%\usepackage[utopia]{mathdesign}
%\usepackage{fourier}

% ----------------------------------------------------------- fancy header stuff
\usepackage{fancyhdr}
\pagestyle{fancy}
\fancyhead[LO,LE]{Math 112L-06}
\chead{\textbf{Strategy for Series I}}
\fancyhead[RO,RE]{Spring 2017}

% ------------------------------------------------------------------- tikz stuff
\usepackage{tikz}
%\usetikzlibrary{decorations.markings}

% ----------------------------------------------------------------------- macros
\newcommand{\RR}{\mathbb{R}}
\newcommand{\QQ}{\mathbb{Q}}
\newcommand{\ZZ}{\mathbb{Z}}
\newcommand{\NN}{\mathbb{N}}
%\DeclareMathOperator{\Hom}{Hom}

% --------------------------------------------------------------- begin document
\begin{document}

\section{\emph{n}th Term Test}

\subsection{Prerequisites?}

None. Can be used for any series you come across.

\subsection{When to use?}

Every. Single. Time.\\

\noindent Unless you have a geometric/$p$-/telescoping series, this should
\emph{always} be the first thing you try. It doesn't always work, but when it
does, you're done with the problem in $<3$ minutes. Shakespeare once said that
there's no greater pain in the world than spending 20 minutes trying 5 different
tests before realizing that all you needed was an $n$th term test to conclude
that the series diverges. It's true. I asked him.\\

\newpage
\section{Comparison Test}

\subsection{Prerequisites?}

All terms must be positive or zero, i.e., $a_n \geq 0$.

\subsection{When to use?}

When there are ugly functions you need to bound. Here are some common ones:
\begin{itemize}

  \item $-1 \leq \sin(\text{anything}) \leq 1$
  \item $0 \leq |\sin(\text{anything})| \leq 1$
  \item $0 \leq \sin^2(\text{anything}) \leq 1$
  \item $0 \leq \arctan(\text{any positive number}) \leq \displaystyle\frac\pi2$
  \item $0 \leq e^{\text{any negative number}} \leq 1$

\end{itemize}
Depending on the context of the question, you might have to come up with sharper
bounds on your own.

\subsection{Why is it useful?}

It replaces an ugly expression ($\sin(?)$, $\cos(?)$, $\arctan(?)$, $e^?$,
etc.), which is impossible to work with, with a number, which is easier to work
with.

\subsection{Examples}

\begin{enumerate}

  \item $\displaystyle \sum_{n=1}^\infty \frac{\arctan(n^2)}{n^2} \gets$ Help, I
  don't know what to do!

  Bound the ugly thing: $\displaystyle \frac0{n^2} \leq \frac{\arctan(n^2)}{n^2} \leq
  \frac{\pi/2}{n^2}$.

  Therefore: $\displaystyle 0 \leq \displaystyle \sum_{n=1}^\infty
  \frac{\arctan(n^2)}{n^2} \leq
  \fbox{\displaystyle\sum_{n=1}^\infty\frac{\pi/2}{n^2}} \gets$ Oh, wait, this
  is just $\displaystyle \frac\pi2\sum_{n=1}^\infty\frac1{n^2}$. I know what to
  do!

  \item $\displaystyle \sum_{n=1}^\infty \frac{10+8\sin(n^6+4)-e^{-\sqrt{e^n
  }}}{n^2} \gets$ lolwut

  What do you do? Bound the ugly thing! Two obvious inequalities: \[
    -8 \leq 8\sin(\text{whatever}) \leq 8 \qquad \text{and} \qquad
    0 \leq e^{\text{anything negative}} \leq 1.
  \]
  Now add these guys up:
  \begin{center}\begin{array}{rrcccl}
    &10 & \leq & 10 & \leq & 10 \\
    &-8 & \leq & 8\sin(n^6+4) & \leq & 8 \\
    (+) \;&-1 & \leq & -e^{-\sqrt{e^n} & \leq & 0 \\\hline
    &1 & \leq & \text{numerator} & \leq & 18
  \end{array}\end{center}

  Therefore: $\displaystyle 0 \leq \fbox{\displaystyle \sum_{n=1}^\infty
  \frac{1}{n^2} \leq
  \sum_{n=1}^\infty\frac{10+8\sin(n^6+4)-e^{-\sqrt{e^n}}}{n^2} \leq
  \sum_{n=1}^\infty \frac{18}{n^2}}\gets$ Not so scary anymore.

  \item $\displaystyle \sum_{n=1}^\infty \frac{10+8\sin(n^6+4)-e^{-\sqrt{e^n
  }}}{\sqrt n} \gets$ Now you try this.

\end{enumerate}

\subsection{Tricky cases}

\begin{enumerate}

  \item $\displaystyle \sum_{n=1}^\infty \frac{\arctan(n)}{n} \gets$
  Superficially, looks very similar to the first example. Secretly, very
  different.

  Bound the ugly thing: $\displaystyle 0 \leq \frac{\arctan(n)}{n} \leq
  \frac{\pi/2}{n}$.

  Therefore: $\displaystyle 0 \leq \displaystyle \sum_{n=1}^\infty
  \frac{\arctan(n)}{n} \leq
  \fbox{\displaystyle\sum_{n=1}^\infty\frac{\pi/2}{n}} \gets$ Oh, wait, this
  diverges. No conclusion :(

  What else can I try? Bound it the other way: $\displaystyle 0 \leq
  \fbox{\displaystyle \frac{\pi/4}{n}
  \leq \frac{\arctan(n)}n} \gets$ Think: Why is this true?

  Therefore: $\displaystyle 0 \leq \displaystyle \sum_{n=1}^\infty
  \frac{\pi/4}{n} \leq
  \sum_{n=1}^\infty\frac{\arctan(n)}{n} \gets$ Yay, thing in the middle
  diverges, so it's conclusive!

\end{enumerate}

\newpage
\section{Limit Comparison Test}

\subsection{Prerequisites?}

All terms must be positive or zero, i.e., $a_n \geq 0$.

\subsection{When to use?}

When there are dominant terms and/or when you have a series that ``looks like''
a geometric series or a $p$-series.

\subsection{Why is it useful?}

We know the precise conditions under which geometric series and $p$-series
converge/diverge. Things that ``look like'' the things we know (usually) behave
like the things we know.

\subsection{Examples}

\begin{enumerate}

  \item $\displaystyle \sum_{n=1}^\infty \frac{1+2^n}{3^n+n^4}$

  Dominant term upstairs: $2^n$. Dominant term downstairs: $3^n$. Do LCT with
  $b_n = \displaystyle \frac{2^n}{3^n}$.

  \item $\displaystyle \sum_{n=1}^\infty \frac{n^2 + 3n + 4}{5n^6-7}$

  Dominant term upstairs: $n^2$. Dominant term downstairs: $5n^6$. Do LCT with
  $b_n = \displaystyle \frac{n^2}{5n^6}$. (You could do it with $b_n =
  \displaystyle \frac{n^2}{n^6}$ as well; coefficients don't really matter.)

  \item $\displaystyle \sum_{n=1}^\infty \frac{5n^6-7}{n^2 + 3n + 4}$

  Dominant term upstairs: $5n^6$. Dominant term downstairs: $n^2$.
  \color{red}Don't waste your time doing LCT. It's still going to work, of
  course, but it's much faster to use the $n$th term test. \color{black} Think:
  Why doesn't the $n$th term test work for the previous two examples?

\end{enumerate}

\newpage
\subsection{Hail Mary}

For mathematical reasons that I'm only willing to explain if you ask me in
person, when you're stuck on a problem and there happens to be a $\displaystyle
\frac1n$ somewhere in the question, you can try doing LCT with $\displaystyle
b_n = \frac1n$ even if it doesn't look like there's any reason it should work.
\\

\noindent Here's an example: $\displaystyle \sum_{n=1}^\infty
\arctan\left(\frac1n\right)$. This is not geometric; not a $p$-series; not
telescoping; the $n$th term test is inconclusive; we can't really bound the ugly
thing with anything useful; there's no dominant term; we don't know what the
integral of $\arctan$ is. Time for a Hail Mary: \[
  \lim_{n\to\infty}\frac{\displaystyle\arctan\left(\frac1n\right)}{\displaystyle\frac1n}
  =\lim_{x\to\infty}\frac{\displaystyle\arctan\left(\frac1x\right)}{\displaystyle\frac1x}
  \stackrel{H}{=}\lim_{x\to\infty}\frac{\displaystyle\left(\frac1{1+\frac1{x^2}}\right)\left(-\frac1{x^2}\right)}
  {\displaystyle-\frac1{x^2}}
  =\lim_{x\to\infty}\frac1{1+\displaystyle\frac1{x^2}}=1,
\]
which is a positive number, so $\displaystyle\sum_{n=1}^\infty
\arctan\left(\frac1n\right)$ and $\displaystyle\sum_{n=1}^\infty \frac1n$ have
the same behavior, i.e., diverge to $+\infty$.

\newpage
\section{Absolute Convergence Theorem}

\subsection{Prerequisites?}

None.

\subsection{When to use?}

When your series has negative terms in it.

\subsection{Why is it useful?}

The CT and LCT can only be used when your series has positive terms (go back and
read their prerequisites again). 90\% of the time, this is how the ACT is used
in a real problem:
\begin{enumerate}

  \item Question gives you $\fbox{\displaystyle \sum_{n=1}^\infty a_n} \gets$ has negative terms; can't
  use CT or LCT :(

  \item You don't answer the question just yet.

  \item Instead, you look at a different series:
  $\fbox{\displaystyle \sum_{n=1}^\infty \left| a_n \right|} \gets$ no
  negative terms; can use CT or LCT :)

  \item If the new series converges, then the ACT tells you that the original
  series converges as well. If the new series diverges, then the ACT is
  inconclusive and you have to try something else.

\end{enumerate}
Please take your time to really understand how the logic of this argument goes.
It's not as straightforward as some of the mathematical arguments you're used
to.

\newpage
\section{All At Once}

Example: $\displaystyle \sum_{n=2}^\infty
\frac{3\cos(4n+5)-2}{n\sqrt[3]{n^2-1}}$.

\begin{itemize}
  \item Don't panic. Be patient. Get rid of the big ugly upstairs first.
  Start off with \[
    -3 \leq 3\cos(4n+5) \leq 3.
  %Subtract $2$ from everything\dots \[
    %-5 \leq 3\cos(4n+5)-2 \leq 1.
  \]
  Subtract $2$ and divide everything by $n \sqrt[3]{n^2-1}$: \[
    \frac{-5}{n\sqrt[3]{n^2-1}} \leq \frac{3\cos(4n+5)-2}{n\sqrt[3]{n^2-1}} \leq
    \frac1{n\sqrt[3]{n^2-1}}.
  \]

  \item The lower bound is negative, which means we might have negative terms in
  our series, so we look at a different series: \[
    \sum_{n=1}^\infty\left|\frac{3\cos(4n+5)-2}{n\sqrt[3]{n^2-1}}\right|.
  \]
  \color{gray}We hope that this new guy converges, because if it diverges, then
  no conclusion can be drawn (why?) and we have to start all over again and try
  something else.\color{black}

  \item Based on our bounds above, this must be true: \[
    0 \leq \left|\frac{3\cos(4n+5)-2}{n\sqrt[3]{n^2-1}}\right| \leq
    \frac5{n\sqrt[3]{n^2-1}}.
  \]
  If you don't understand how I got this, \emph{please ask!} But I want you
  to think about it for a minute. 
  
  \item Thus, \[
    0 \leq \sum_{n=2}^\infty \left|\frac{3\cos(4n+5)-2}{n\sqrt[3]{n^2-1}}\right| \leq
    \sum_{n=2}^\infty\frac5{n\sqrt[3]{n^2-1}}.
  \]

  \item Now the thing on the right has a dominant term downstairs, which is
  $n\sqrt[3]{n^2} = n\cdot n^{2/3}=n^{5/3}$. In other words, \[
    \sum_{n=2}^\infty\frac5{n\sqrt[3]{n^2-1}} \qquad\text{``looks like''}\qquad
    \sum_{n=2}^\infty\frac5{n^{5/3}}.
  \]

  \item Now we're good to go:\begin{itemize}

    \item By $p$-series, $\displaystyle \sum_{n=2}^\infty \frac5{n^{5/3}}$
    converges because $p=\displaystyle\frac53>1$.

    \item Therefore, by LCT, $\displaystyle\sum_{n=2}^\infty\frac5{n\sqrt[3]{n^2-1}}$
    converges as well. (You actually have to do the math to find
    $\displaystyle\lim_{n\to\infty}\frac{a_n}{b_n}$; I'm not doing it here.)

    \item Therefore, by CT, $\displaystyle\sum_{n=2}^\infty
    \left|\frac{3\cos(4n+5)-2}{n\sqrt[3]{n^2-1}}\right|$ converges as well.

    \item Therefore, by ACT, $\displaystyle\sum_{n=2}^\infty
    \frac{3\cos(4n+5)-2}{n\sqrt[3]{n^2-1}}$ converges as well.

  \end{itemize}
\end{itemize}

\end{document}
