\documentclass[11pt]{article}

% ----------------------------------------------------------- essential packages
\usepackage[margin=1in]{geometry}
\usepackage{amsmath}
%\usepackage{amsfonts}
\usepackage{amsthm}
\usepackage{amssymb}
\usepackage{booktabs} % somewhat essential, I guess
\usepackage{setspace} % for double spacing, etc.

% ------------------------------------------------------------------- font stuff
\renewcommand{\rmdefault}{ptm}
\usepackage[lite,zswash,eucal]{mtpro2}
%\usepackage[charter]{mathdesign}
%\usepackage[utopia]{mathdesign}
%\usepackage{fourier}

% ----------------------------------------------------------- fancy header stuff
\usepackage{fancyhdr}
\pagestyle{fancy}
\fancyhead[LO,LE]{Math 112L-06}
\chead{\textbf{Series Practice II}}
\fancyhead[RO,RE]{Spring 2017}

% ------------------------------------------------------------------- tikz stuff
\usepackage{tikz}
%\usetikzlibrary{decorations.markings}

% ----------------------------------------------------------------------- macros
\newcommand{\RR}{\mathbb{R}}
\newcommand{\QQ}{\mathbb{Q}}
\newcommand{\ZZ}{\mathbb{Z}}
\newcommand{\NN}{\mathbb{N}}
%\DeclareMathOperator{\Hom}{Hom}

% --------------------------------------------------------------- begin document
\begin{document}

\begin{enumerate}

  \item Suppose the terms $a_n$ are positive and decreasing. Let $f(x)$ be a
  continuous decreasing function such that $f(n) = a_n$. (The last sentence is
  just fancy math-speak for ``$f(x)$ is the continuous version of $a_n$.'' I'm
  just writing it here so you know what it means when you see it somewhere
  else.) \begin{enumerate}

    \item Order these 6 quantities from smallest to biggest: \[
      \sum_{n=7}^\infty a_n
      \qquad
      \sum_{n=8}^\infty a_n
      \qquad
      \sum_{n=9}^\infty a_n
      \qquad
      \int_7^\infty f(x) \, dx
      \qquad
      \int_8^\infty f(x) \, dx
      \qquad
      \int_9^\infty f(x) \, dx
    \]

    \item Fill in the following blanks: \[
      \sum_{n=1234}^{5678} a_n <
      \int_{\,\fbox{\phantom{n}}}^{\fbox{\phantom{n}}} f(x) \, dx
      \qquad\qquad\qquad
      \sum_{n=1234}^{5678} a_n >
      \int_{\,\fbox{\phantom{n}}}^{\fbox{\phantom{n}}} f(x) \, dx
    \] \[
      \int_{1234}^{5678} f(x) \, dx <
      \sum_{n=\,\fbox{\phantom{n}}}^{\fbox{\phantom{n}}} a_n
      \qquad\qquad\qquad
      \int_{1234}^{5678} f(x) \, dx >
      \sum_{n=\,\fbox{\phantom{n}}}^{\fbox{\phantom{n}}} a_n
    \]

    \item Fill in the following blanks: \[
      \sum_{n=112}^{2017} a_n =
      \sum_{n=112}^{1776} a_n +
      \sum_{n=\,\fbox{\phantom{n}}}^{\fbox{\phantom{n}}} a_n
    \] \[
      \int_{112}^{2017} f(x) \, dx =
      \int_{112}^{1776} f(x) \, dx +
      \int_{\,\fbox{\phantom{n}}}^{\fbox{\phantom{n}}} f(x) \, dx
    \]

  \end{enumerate}

  \item Decide if they converge or diverge: \begin{enumerate}

    \item $\displaystyle \sum_{n=1}^\infty 2^{-n}$

    \item $\displaystyle \sum_{n=1}^\infty n^{-2}$

    \item $\displaystyle \sum_{n=1}^\infty 0.2^{-n}$

    \item $\displaystyle \sum_{n=1}^\infty n^{-0.2}$

    \item $\displaystyle \sum_{n=1}^\infty
    \left( 2n^{-3} - 4n^{-5} + 6n^{-7} - 8n^{-9} \right)$

    \item $\displaystyle \sum_{n=1}^\infty
    \left( \frac2{n^3} - \frac3{2^n} \right)$

    \item $\displaystyle 
    1 + \frac1{2\sqrt2} + \frac1{3\sqrt3} + \frac1{4\sqrt4} + \cdots$

    \item $\displaystyle \sum_{n=1}^\infty
    \frac{n^{-1.5}+1.5\sqrt n}{n^{1.5}}$

  \end{enumerate}

  \item Find all values of $x$ for which $\displaystyle \sum_{n=1}^\infty \left(
  \frac xn - \frac{555}{n+1} \right)$ converges.

  \item Consider the series $\displaystyle \sum_{n=1}^\infty \frac1{(n+1)^2}$.
  \begin{enumerate}
    
    \item Show that it converges.

    \item Find the sum of the first three terms and write it as a fraction.
    Don't use a calculator. You can do this by hand. I have faith in you.

    \item Find lower and upper bounds for the error in using your approximation
    in part (b).
    
    \item Based on your answers to parts (b) and (c), to how many decimal places
    can you accurately determine the value of the series? What is it? Again,
    don't use a calculator. You can do this by hand. I have faith in you.

    \item How many terms do you need to add in order to get an estimate that's
    within 0.0005 of the actual value of the series?

  \end{enumerate}

\end{enumerate}

\end{document}
